\documentclass{article}

\usepackage[margin=0.5in]{geometry}
\usepackage{graphicx}
\usepackage[labelformat=empty]{caption}
\usepackage{csvsimple}
\usepackage{longtable}
\usepackage{flafter}
\usepackage{afterpage}

\pagenumbering{gobble}

\begin{document}

%% Supplemental Figure 1

\begin{figure}
  \begin{center}
    \includegraphics[width=7.5in]{figures/Suppl1.pdf}
  \end{center}
  \caption*{\textbf{Supplementary Fig. 1: Performance of predictors trained on randomly-selected sets of genes plotted as a function of the set size.} Performance was evaluated through leave-pair-out cross-validation and displayed as area under the ROC curve (AUC). The three panels correspond to binary classification tasks comparing early (“A”), intermediate (“B”) and late (“C”) disease stages. The color scheme, as introduced in Fig. 1b, denotes the dataset and brain region (specified as Brodmann Area) of samples used in each analysis.}
\end{figure}

%% Supplemental Figure 2

\begin{figure}
  \begin{center}
    \includegraphics[width=7.25in]{figures/Suppl2.pdf}
  \end{center}
\caption*{\textbf{Supplementary Fig. 2: Concordance of treatment replicates across the two 3’-DGE experiments.} Shown are log-fold change values for all genes that were significantly (FDR $\leq$ 0.05) perturbed in both 3’-DGE experiments. Spearman correlation between the two experiments is displayed in the bottom right corner of each panel.}
\end{figure}

%% Supplemental Figure 3

\begin{figure}
  \begin{center}
    \includegraphics[width=7.5in]{figures/Suppl3.pdf}
  \end{center}
  \caption*{\textbf{Supplementary Fig. 3: Assessment of compound toxicity.} Nuclei counts estimated from microscopy images (x-axis) are plotted against mRNA abundance (y-axis). The mRNA abundance was computed as the total number of transcripts in the post-perturbational gene expression profile of the corresponding compound. Marginal distributions presented on the top and the right-hand side exhibit bi-modality, suggesting natural thresholds for determining compound neurotoxicity. A vertical dashed line is used to classify compounds into Toxic (red) and Non-Toxic (black) categories for Fig. 3.}
\end{figure}

%% Supplemental Figure 4

\begin{figure}
  \begin{center}
    \includegraphics[width=7.25in]{figures/Suppl4.pdf}
  \end{center}
  \caption*{\textbf{Supplementary Fig. 4: Raw expression values of selected interferon-stimulated genes.} Each panel shows normalized transcript counts for a single interferon-stimulated gene (ISG). Individual points correspond to compounds that have a strong (TAS=1) or weak (TAS=2,3) binding to TYK2. Unadjusted $p$-values from two-sided Wilcoxon Rank Sum tests comparing the expression distributions between strong and weak binders are displayed in the top left corner of each panel.}
\end{figure}

%% Supplemental Figure 5

\begin{figure}
  \begin{center}
    \includegraphics[width=7.25in]{figures/Suppl5.pdf}
  \end{center}
  \caption*{\textbf{Supplementary Fig. 5: Pairwise Tanimoto similarity of compounds profiled by 3’ DGE.} Rows and columns of the heatmap are ordered according to hierarchical clustering. The boxplot in the legend shows the distribution of all non-diagonal values (i.e., non-self similarities).}
\end{figure}
%% Supplemental Figure 6

\begin{figure}
  \begin{center}
    \includegraphics[width=7.25in]{figures/Suppl6.pdf}
  \end{center}
  \caption*{\textbf{Supplementary Fig. 6: Comparison of machine learning methods implemented in DRIAD.} Each method was used to train a model to distinguish early (“A”) vs. late (“C”) disease stages from AMP-AD gene expression data, where only genes associated with a given drug perturbation were considered by each predictor. Model performance was evaluated through leave-pair-out cross-validation and reported as Area under the ROC curve (AUC). The boxplots show how performance varies across 68 drug-associated gene lists that had at least 10 genes that were significantly (FDR $\leq$ 0.05) perturbed by the corresponding drug perturbations. The lower, middle and upper hinges of the boxplot correspond to the 25\%, 50\% (i.e., median) and 75\% quantiles, respectively. The whiskers cover observations that are within 1.5 of the interquantile range (IQR) on each side. Ouliers beyond 1.5*IQR are shown as individual points.}
\end{figure}

\afterpage{
\begin{longtable}{l|l|l|l|l|l|l|l}
    \multicolumn{2}{l}{\bfseries Name} & \multicolumn{6}{l}{\bfseries Sequence} \\ \hline
    \multicolumn{2}{l}{E5V6NEXT}  & \multicolumn{6}{l}{5'-iCiGiCACACTCTTTCCCTACACGACGCrGrGrG-3'} \\ \hline
    \multicolumn{2}{l}{E3V6NEXT*} & \multicolumn{6}{l}{/5Biosg/ACACTCTTTCCCTACACGACGCTCTTCCGATCT[BC6]NNNNNNNNNN} \\
    \multicolumn{2}{l}{*BC6 sequences are below} & \multicolumn{6}{l}{TTTTTTTTTTTTTTTTTTTTTTTTTTTTTTVN-3'} \\ \hline
    \multicolumn{2}{l}{SINGV6}    & \multicolumn{6}{l}{5'-/5Biosg/ACACTCTTTCCCTACACGACGC-3'} \\ \hline
    \multicolumn{2}{l}{P5NEXTPT5} & \multicolumn{6}{l}{5'-AATGATACGGCGACCACCGAGATCTACACTCTTTCCCTACACGACGCTC} \\
    \multicolumn{2}{l}{}          & \multicolumn{6}{l}{TTCCG*A*T*C*T-3'} \\ \hline
    \multicolumn{2}{l}{P5\_PCR}   & \multicolumn{6}{l}{AATGATACGGCGACCACCGAG} \\ \hline
    \multicolumn{2}{l}{P7\_PCR}   & \multicolumn{6}{l}{CAAGCAGAAGACGGCATACGAG} \\ \hline
    \multicolumn{2}{l}{N701}      & \multicolumn{6}{l}{5' CAAGCAGAAGACGGCATACGAGATTCGCCTTAGTCTCGTGGGCTCGG} \\ \hline
    \multicolumn{2}{l}{N703}      & \multicolumn{6}{l}{5' CAAGCAGAAGACGGCATACGAGATTTCTGCCTGTCTCGTGGGCTCGG} \\ \hline
    \bfseries Name & \bfseries Sequence & \bfseries Name & \bfseries Sequence & \bfseries Name & \bfseries Sequence & \bfseries Name & \bfseries Sequence
    \csvreader[head to column names]{primers.csv}{}
    {\\ \hline \csvcoli&\csvcolii&\csvcoliii&\csvcoliv&\csvcolv&\csvcolvi&\csvcolvii&\csvcolviii}
    \\ \hline
    \caption*{\textbf{Supplementary Table 1: The complete list of primers used for 3' Digital Gene Expression.}}
\end{longtable}}

\end{document}
